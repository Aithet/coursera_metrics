\documentclass[12pt,a4paper]{article}
\usepackage[utf8]{inputenc}
\usepackage[russian]{babel}

\usepackage{amsmath}
\usepackage{amsfonts}
\usepackage{amssymb}
\usepackage{url}
\usepackage[left=2cm,right=2cm,top=2cm,bottom=2cm]{geometry}

\renewcommand{\P}{P}

\begin{document}

2 1 1 Условное математическое ожидание. Определение

\url{http://youtu.be/ldbqnBKwtto}

ок

2 1 2 Условное математическое ожидание. Пример подсчёта

\url{http://youtu.be/KUglGmmeXHE}

0:18 исправить опечатку в синей строке внизу, написано <<Пример подсТчёта>>

4:14 тусклый маркер, местами ужасно тусклый

2 1 3 Условная дисперсия. Пример подсчёта

\url{http://youtu.be/gpAemI7ODVI}

0:26 ошибка в формуле. В интеграле вместо $dx$ должно быть $ds$

1:19 вместо <<список свойств>> лучше написать <<Свойства условного математического ожидания>>

добавить небольшим шрифтом сверху:

a, b --- константы

s, r --- случайные величины

Сначала появляется только $E(as+b|r)=aE(s|r)+b$

в 1:31 дополнительно появляется $E(E(s|r))=E(s)$

2:48-2:52 --- удалить оговорку

текст должен быть:

<<Аналогично определяется и условная ковариация>>

он звучит дальше, просто вырезать 4 секунды с оговоркой

3:12 уточнить название сверху <<Свойства условной дисперсии>>

добавить небольшим шрифтом сверху:

$a, b$ --- константы

$s, r$ --- случайные величины

4:02---до конца (весь пример расчета) - вырезать и вставить в момент времени 2:56

6:30, 7:07 --- очень тусклый маркер

2 1 4 Геометрическая иллюстрация условного математического ожидания

\url{http://youtu.be/HsxokLgBWUw}

отрезать и полностью убрать начало вплоть до 1:23

2:25 тусклый маркер

3:50 тусклый маркер

7:18 тусклый маркер

8:10 он же такой же

2.1.5. Условная дисперсия МНК оценок. 

\url{http://youtu.be/sUnzxnE9pqs}

0:33 немного поменять подписи:

- Дисперсия, $Var(r)$ --- квадрат длины случайной величины r

- Корреляция, $Corr(r,s)$ --- косинус угла между величинами r и s

2:01 ковариационная матрица вектора $\varepsilon$ --- сделать букву эпсилон синей как во всех формулах

3:42-3:52: разместить формулы 2 и 3 чуть правее формулы 1 (возможно не с кружочками, а с черточками перед ними)

по сути формулы 2 и 3 поясняют формулу 1 и формула 1 полностью их заменяет

3:52: добавить (с таким же отступом как формулу 1) формулу $E(\varepsilon_i |X)=0$

4:20 внизу вставить <<$|X$>> в формулы, то есть должно быть:

$Var(\hat{\beta}_j|X), Cov(\hat{\beta}_j,\hat{\beta}_l|X)$

4:27 Формулы и предпосылки исчезают, появляется:

<<Для парной регрессии:

$Var(\hat{\beta}_1)=\sigma^2\frac{\sum x_i^2}{n\sum (x_i-\bar{x})^2}$

$Var(\hat{\beta}_2)=\frac{\sigma^2}{\sum (x_i-\bar{x})^2}$

$Cov(\hat{\beta}_1, \hat{\beta}_2)=\sigma^2\frac{-\bar{x}}{\sum (x_i-\bar{x})^2}$>>

4:37 --- до конца ---- отрезать в отдельное видео. Этот фрагмент назвать <<Условная дисперсия МНК оценок. Начало доказательства>>


6:30 и далее тусклый маркер


2 1 6 Условная дисперсия МНК оценок. Завершение доказательства.

\url{http://youtu.be/nQa3JuLHIu8}

0:20 --- 0:50 предполагалось ускорение видео, я специально там молчал. Или от этой идеи отказались почему-то?

2 1 7 Дисперсии оценок в общем виде

\url{http://youtu.be/Sbtm_aMCgS0}

1:42 заменить ответ на

--- При случайных регрессорах безусловная дисперсия считается слишком сложно...

3:09 убрать те надписи, что есть и заменить их на:

Теорема:

(формула остается без изменений)

3:41 дополнительно появляется:

Свойства дисперсии:

$Var(a\cdot x_i)=a^2 Var(x_i)$

3:55 появляется только формула, а слово <<свойство>> перед ней убираем

4:10 вместо <<напомним что>> появляется <<Свойства транспонирования:>>





2 1 8 Доказательство формулы для ковариационной матрицы

\url{http://youtu.be/1GG46RDfZSs}

2 1 9 Оценка ковариационной матрицы и доверительный интервал для коэффициента

\url{http://youtu.be/GFx6vJm7MQI}

0:26 Как оценить $\sigma^2$? --- поставить знак вопроса в конце заголовка

1:48 Поставить запятую после <<А именно,>>

2:22 под корнем пропущено $Var$ с крышкой. То есть должно быть:

$se(\hat{\beta}_j)=\sqrt{\widehat{Var}(\hat{\beta}_j)}$

3:16 заменить <<ЛИНАЛ>> на <<В общем виде:>>

3:40---до конца начало доски  --- отрезать и вставить в 5:55 фрагмента 2.1.11


2 1 10 Статистические свойства оценок коэффициентов

\url{http://youtu.be/C9-N957ZORY}

0:25 убрать сокращение <<БСХС>> (и везде далее убрать БСХС), оставить только <<Большой Список Хороших Свойств>> 


1:44 убрать <<БСХС>> оставить <<Предпосылки>>

2:30 пропущена запятая после слова константа и нижний индекс $i$ у игрека, должно быть:

<<С помощью МНК оценивается регрессия $y_i$ на константу, $x_i$ и $z_i$>>

%3:04 оценка $\hat{\beta}$ в матричном виде???

3:15. Убрать <<БСХС>>, оставить <<Предположения на $\varepsilon_i$>>

3:41 между строчками <<$E(e_i^2 | \text{ все регрессоры })$>> и <<В матричном виде>> вставить строку:

<<Или: $Var(\varepsilon_i | \text{  все регрессоры })=\sigma^2$>>

3:53 выше формулы добавить <<Условная некоррелированность:>>

4:21 убрать <<БСХС>>, оставить заголовок <<Предпосылки на регрессоры:>>

4:21 <<Векторы отдельных наблюдений...>> должно появляться в 4:48

4:48 <<С вероятностью 1...>> должно появляться в 4:21

6:02--6:17   Вырезать этот кусок и вставить в момент 4:46

6:02 Во фразе добавить <<:>> и слово <<существует>>. Сделать одинаковый отступ с пунктом <<С вероятностью 1 среди регрессоров нет линейно зависимых>>. Чтобы  этот пункт выглядел примерно так:

* С вероятностью 1 среди регрессоров нет линейно зависимых.

  Синонимы в матричном виде: $rk(X'X)=k$, $(X'X)^{-1}$ существует или $det(X'X)\neq 0$
  
  
6:35, долой <<БСХС>>, оставляем <<Базовые свойства (теорема Гаусса-Маркова)>>

7:56 заменяем на 

* Оценки несмещены:

условно, $E(\hat{\beta}_j |X)=\beta_j$

и безусловно, $E(\hat{\beta}_j)=\beta_j$

10:05 меняем заголовок на <<Базовые свойства:>>

10:51 к последней формуле добавляем запятую в конце и ниже пишем

<<где $\hat{\sigma}^2=RSS/(n-k)$>>

11:25 оставляем <<Асимптотические свойства:>> без БСХС

11:35 дописываем в конце <<... по вероятности, т.е. $\hat{\beta}_j$ состоятельны>>

12:35 заголовок без бсхс, <<При нормальности $\varepsilon_i$:>>

в формулу добавляем <<$|X$>>, т.е:

Если дополнительно известно, что $\varepsilon_i | X \sim N(0,\sigma^2)$


2.1.11 Построение доверительных интервалов и проверка гипотез

\url{http://youtu.be/a36WGysGPnA}

0:57 появляется формула 

$\frac{\hat{\beta}_j-\beta_j}{se(\hat{\beta}_j)} \to N(0,1)$

1:25 появляется текст:

Проверять гипотезы можно в двух случаях:

* Число наблюдений велико

* Случайные ошибки нормальны

1:48 заменяем фразу <<Возможно строить в двух подходах>> на <<Проверка гипотезы о коэффициенте $\beta_j$:>>

3:03 <<Проверяемая гипотеза $H_0$>>

3:16 вместо дописывания <<против $H_a$>> делаем новый пункт <<Конкурирующая гипотеза $H_a$>>

4:01 Стираем старые строки и делаем новый заголовок <<Практические шаги:>>

4:05 добавляем под заголовком

1. Формулируем гипотезу $H_0$ и выбираем уровень значимости $\alpha=\P( \text{ отвергнуть }H_0|H_0 \text{ верна })$

4:33 добавляем еще пункт

2. Рассчитываем наблюдаемое значение тестовой статистики, $S_{obs}$

4:40 добавляем еще пункт

3. Находим критическое значение тестовой статистики, $S_{cr}$

4:47 добавляем еще пункт

4а. Сравниваем $S_{obs}$ и $S_{cr}$, делаем вывод об $H_0$

5:25 добавляем еще пункт:

4б. Сравниваем $P$-значение и $\alpha$, делаем вывод об $H_0$

5:55 сюда вставляется отрезанный кусок от 2.1.9

текущий фрагмент 5:55---11:23 удаляется!!! так как он уже идет во фрагменте 2.1.12!!!


2.1.12 Доверительный интервал для $\sigma^2$

\url{http://youtu.be/llSCwxEUjNw}

ок

2.1.13 Проверка гипотез о $\beta_j$

\url{http://youtu.be/belwLt1rBiY}

6:30 заменяем заголовок <<Описание любого теста>> на <<Распространенная форма записи:>>


2.1.14 Интерпретация стандартной таблички 

\url{http://youtu.be/FwqwO3E1NEA}

пожелание: растянуть табличку на сколько можно, много не получится, но всё же

2.1.15. Особенности проверки гипотез

\url{http://youtu.be/CfMEeFb8g5k}

7:56 немного подредактировать пункты:

* Асимптотически: $N(0,1)$

* При нормальности $\varepsilon_i$: $t_{n-k}$

2.1.16 Гипотеза о линейном ограничении 

\url{http://youtu.be/WSjoAeujXe4}

вырезать собирание с мыслями 7:04 - 7:32

9:58 маленький кусочек не отражен зеркально
 

2 2 1 Работа со случайными величинами в R 

\url{http://youtu.be/92v1Br60Ys4}

ок

2 2 2 Проверка гипотез о коэффициентах

\url{http://youtu.be/a27n8-DZVNQ}

удалить кусок 3:51--4:03 полностью 

удалить кусок 5:32--5:43 (тишина)


2 2 3 Стандартизированные коэффициенты

\url{http://youtu.be/qvMbDVrSBWI}

кусок 3:41 - 8:55 удаляем полностью
 
2 2 4  Сохранение и загрузка данных 

\url{http://youtu.be/8chddRL-EK4}

10:48 --- до конца --- удалить

2 2 5 Загрузка данных RLMS

\url{http://youtu.be/DYfk64xhl1U}

надо переснять полностью

\end{document}