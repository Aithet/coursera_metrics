\documentclass[12pt,a4paper]{article}
\usepackage[utf8]{inputenc}
\usepackage[russian]{babel}
\usepackage[OT1]{fontenc}
\usepackage{amsmath}
\usepackage{amsfonts}
\usepackage{amssymb}
\usepackage{url}
\usepackage[left=2cm,right=2cm,top=2cm,bottom=2cm]{geometry}
\newcommand{\hb}{\hat{\beta}}


\def \useR{$[$R$]$ }

%% эконометрические сокращения
\def \hb{\hat{\beta}}
\def \hs{\hat{\sigma}}
\def \hy{\hat{y}}
\def \hY{\hat{Y}}
\def \v1{\vec{1}}
\def \e{\varepsilon}
\def \he{\hat{\e}}
\def \z{z}
\def \hVar{\widehat{\Var}}
\def \hCorr{\widehat{\Corr}}
\def \hCov{\widehat{\Cov}}


%% лаг
\renewcommand{\L}{\mathrm{L}}

%% алая и белая розы
%% запускается так: \WhiteRose[масштаб], например, \WhiteRose[0.5]
\newcommand{\WhiteRose}[1]{\begingroup
\setbox0=\hbox{\includegraphics[scale=#1]{/home/boris/science/econometrix/em301/roses/Yorkshire_rose.pdf}}%
\parbox{\wd0}{\box0}\endgroup}

\newcommand{\RedRose}[1]{\begingroup
\setbox0=\hbox{\includegraphics[scale=#1]{/home/boris/science/econometrix/em301/roses/Lancashire_rose.pdf}}%
\parbox{\wd0}{\box0}\endgroup}

\newcommand{\WhiteRoseLine}{
\begin{center}
\WhiteRose{0.3} Версия Белой Розы \WhiteRose{0.3}
\end{center}}

\newcommand{\RedRoseLine}{
\begin{center}
\RedRose{0.3} Версия Алой Розы \RedRose{0.3}
\end{center}}



\begin{document}
1 глава (повторный отсмотр)

1-1-1 Суть метода наименьших квадратов

\url{http://youtu.be/UGD_u4tpZQU}

0:52 сейчас в этот момент появляется надпись <<* временные ряды>> --- убираем

1:05 сейчас в этот момент появляется таблица --- убираем

1:06 появляется список:

* временные ряды

* перекрестные данные

* панельные данные

1:17 появляется <<Временные ряды>> и ниже табличка (которая была в 1:05)

над табличкой заголовок: <<Данные по России>>

3:24 во фразе <<по каждой переменной $n$ наблюдений $y_1, y_2, ..., y_n$>> сделать перенос

строки после слова <<переменной>>, т.е.

<<по каждой переменной

n наблюдений: $y_1, y_2, \ldots, y_n$>>

и <<$n$>> сделать синим цветом

7:25 сделать букву <<$Q$>> синим цветом

1-1-2 изменить название на <<Пример 1. Регрессия на константу.>>

\url{http://youtu.be/qS66GY0bDR0}


1-1-3 изменить название на <<Пример 2. Парная регрессия. Начало.>>

\url{http://youtu.be/TNodFvAHxDA}


1-1-4 изменить название на <<Пример 2. Парная регрессия. Окончание.>>

\url{http://youtu.be/0NHwfP6cvTU}


1-1-5 изменить название на <<МНК на графике. Случай множества регрессоров.>>

\url{http://youtu.be/SB1TMb8Si9U}

10:52 вставить синим цветом $\hb_1$, $\hb_2$, $\hb_3$ во фразу над системой уравнений, чтобы получилось

<<Оценки $\hb_1$, $\hb_2$, $\hb_3$ находятся из системы>>

1-1-6 Ликбез по линейной алгебре

\url{http://youtu.be/TieRRpVBFy4}

0:31---1:07 сделать выезжающие надписи пунктами, то есть

0:31 добавить кружочек (буллет) перед <<Сумма квадратов остатков>>

0:44 добавить кружочек (буллет) перед <<Общая сумма квадратов>>

1:07 добавить кружочек (буллет) перед <<Объясненная сумма квадратов>>

чтобы выглядело:

* Сумма квадратов остатков

...

* Общая сумма квадратов

...

* Объясненная сумма квадратов


5:06 сделать <<т.к.>> черным цветом. Общий принцип: синий --- для формул, черный для текста

5:24 сделать <<т.к.>> черным цветом. Общий принцип: синий --- для формул, черный для текста


1-1-7 изменить название на <<Геометрическая иллюстрация регрессии на константу>>

\url{http://youtu.be/TxaiEPyaCdU}



1-1-8 исправить название на <<Геометрическая иллюстрация МНК для множества регрессОров>>

(сейчас опечатка в последнем слове) 

\url{http://youtu.be/9Iw9spYTCNM}

0:27 исправить формулу под словом <<Модель>> на $y_i=\beta_1+\beta_2 x_i +\beta_3 z_i + 
\varepsilon_i$


1-1-9 Коэффициент детерминации

\url{http://youtu.be/gjczkpnxd3E}

1:32 вместо <<доля объясненного разброса в общем разбросе>> должно быть <<сумма квадратов 
остатков>>

1:43 дополнительно (ниже уже написанного) появляется формула $R^2=ESS/TSS$



1-1-10 Мораль первой лекции 

\url{http://youtu.be/7Qniqa3e_2g}

1-2-1 Консольный режим в R

\url{http://youtu.be/DKgFFjJjpjI}


1-2-2 Написание первого скрипта в R

\url{http://youtu.be/6eIIEgemYh8}

1-2-3 Установка пакетов в R Получение справки

\url{http://youtu.be/T2Drwb0FZis}

1-2-4 Первый взгляд на набор данных

\url{http://youtu.be/WozZC2XR1ro}

1-2-5 Метод наименьших квадратов. Пример с машинами

\url{http://youtu.be/Hq_sbkNdxFk}

1-2-6 Метод наименьших квадратов. Пример с фертильностью

\url{http://youtu.be/6vmJAjWrgl4}

\end{document}