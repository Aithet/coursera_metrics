\documentclass[12pt,a4paper]{article}
\usepackage[utf8]{inputenc}
\usepackage[russian]{babel}

\usepackage{amsmath}
\usepackage{amsfonts}
\usepackage{amssymb}
\usepackage{url}
\usepackage[left=2cm,right=2cm,top=2cm,bottom=2cm]{geometry}
\begin{document}
Комменты по исправленной 1:

1 1 8 Геометрическая иллюстрация МНК для множества регрессеров 

\url{http://youtu.be/40IXgQN1CC8}

0:28 Добавить сверху: (по прежнему не исправлено)

Модель:

$y_i=\beta_1+\beta_2 x_i+\beta_3 z_i + \varepsilon_i$


1 1 9 Коэффициент детерминации

\url{http://youtu.be/Sag4ZnIlguY}

1:30 Появляется:

В моделях со свободным членом:

$TSS=ESS+RSS$

$TSS=\sum (y_i-\bar{y})^2$ --- общий разброс $y$

$ESS=\sum (\hat{y}_i-\bar{y})^2$ --- объяснённый разброс $y$

$RSS=\sum \hat{\varepsilon}_i^2$ --- сумма квадратов ошибок

(более крупным шрифтом):

$R^2=ESS/TSS$

1:58 на экране:

Коэффициент детерминации, $R^2 \in [0;1]$

(более крупным шрифтом):

$R^2=ESS/TSS$

* доля объяснённого разброса $y$ в общем разбросе $y$


2:33 исправить на:

Теорема:

Если в регрессию включён свободный \\
член ($y_i=\beta_1 + \ldots$) и оценки МНК \\
единственны, то $R^2$ равен квадрату \\
выборочной корреляции между $y$ и $\hat{y}$

2:41 выезжает формула (осталось неисправленным)

\[
R^2=sCorr^2(y,\hat{y})
=\left(\frac{\sum (y_i-\bar{y})(\hat{y}_i-\bar{y})}{\sqrt{\sum(y_i-\bar{y})^2}\sqrt{\sum(\hat{y}_i-\bar{y})^2}}\right)^2
\]

1 1 5 Метод наименьших квадратов

\url{http://youtu.be/pacc_MRwzhw}

если я правильно понимаю общий стиль --- все формулы синим, тогда: 

0:29 --- формулу $y_i=\beta + \varepsilon_i$ в синий цвет перекрасить

0:39 --- формулу $y_i=\beta_1 +\beta_2 x_i + \varepsilon_i$ в синий цвет перекрасить

1:06 добавить внизу синюю формулу:  (повторная просьба)

$\hat{\beta}_1=\bar{y}-\hat{\beta}_2 \bar{x}$

1 1 7 Геометрическая иллюстрация МНК без регрессеров

\url{http://youtu.be/GGWZwRIAc2o}

1 1 4 исправить название фрагмента на <<МНК для парной регрессии  --- окончание>>

\url{http://youtu.be/4o8R0mQ3dhw}

1 1 2 исправить название фрагмента на  <<МНК для парной регрессии --- начало>>

Здесь ошибка в номере! Это должен быть 1 1 3!

\url{http://youtu.be/FWKG0gHNuhE}

 

\end{document}