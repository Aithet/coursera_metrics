\documentclass[12pt,a4paper]{article}
\usepackage[utf8]{inputenc}
\usepackage[russian]{babel}
\usepackage{url}
\usepackage{amsmath}
\usepackage{amsfonts}
\usepackage{amssymb}
\usepackage[left=1cm,right=1cm,top=1cm,bottom=2cm]{geometry}
\newcommand{\e}{\varepsilon}
\renewcommand{\b}{\beta}
\newcommand{\hb}{\hat{\b}}
\newcommand{\hs}{\hat{\sigma}}
\DeclareMathOperator*{\plim}{plim}
\usepackage{comment}
\usepackage{graphicx}
\begin{document}


10-1-1 Медианная регрессия

\url{https://youtu.be/nOmPHv9dpUk}

0:24 исправить заголовок слайда на <<Три сюжета>>

1:43 во втором пункте исправить слово <<минизации>>, должно быть:

* При минимизации $Q(\hb_1,\hb_2)$ получаем состоятельные оценки $\hb_1$, $\hb_2$


10-1-2 Квантильная регрессия

\url{https://youtu.be/zecZye4a0_8}

2:17 вставить пробел перед <<и>>

* Нет явных формул для оценок коэффициентов и стандартных ошибок

3:47 убрать с графика подписи $\tau=0.1$ и $\tau=0.9$. Вместо них: прямо на графике, на той части, что закрашена в клеточку, написать $0.1$; на той части графика, что закрашена в полосочку, написать $0.8$; на той части, что не закрашена (справа от правой вертикальной линии), написать $0.1$. 

6:54 Сделать название слайда в одну строчку, горизонтальную ось графика вместо totsp назвать <<Общая площадь, м$^2$>>. Вертикальную ось вместо price назвать <<Цена квартиры, тыс. у.е.>>


10-1-3 Алгоритм случайного леса

\url{https://youtu.be/Ld_lhkfQ6bI}

2:05 продублировать справа сверху от дерева ту табличку, что висит в 1:16. И пусть она рядом с деревом висит

10-1-4 Пример построения регрессионного дерева

0:16 изменить название видеофрагмента (внизу на синей полосе) на <<Пример построения регрессионного дерева>>

\url{https://youtu.be/63fozZFGOG0}

10-1-5 Суть байесовского подхода

\url{https://youtu.be/jsEx2Pom-1I}

1:44 у вертикальной оси на графике на верхушке добавить стрелочку вверх; под пересечением вертикальной и горизонтальной осей добавить подпись <<0>> (без кавычек)

2:29 добавить запятую после слова <<Например>>

10-1-6 Расчет апостериорного распределения. Пример 1

\url{https://youtu.be/b5XvY97nMq0} ок

10-1-7 Расчет апостериорного распределения. Пример 2

\url{https://youtu.be/6w4bRrAjDvs} ок

10-1-8 Алгоритм MCMC и логит-модель

\url{https://youtu.be/vJKMxb3BJYw}

2:09 справа внизу вместо <<из $f(y|\theta)$>> должно быть <<из $f(\theta|y)$>> 

7:59 исправить появляющуюся надпись на 

Оценки метода максимального правдоподобия для логит и пробит модели не существуют

10-1-9 Регрессия пик-плато и спасибо 

\url{https://youtu.be/c5BNpk_4o1g}

5:22 выше уравнения добавить надпись: <<Апостериорные средние:>>

7:11 сделать небольшой разрыв между первым предложение и остальным текстом и последним предложением и остальным текстом 

10-2-1 Квантильная регрессия и алгоритм случайного леса

\url{https://youtu.be/hec3_Xu9Bfo} ок

10-2-2 Логит-модель: байесовский подход

\url{https://youtu.be/VoMy8Qt_AVk}

0:16 название видеофрагмента (на синей полосе внизу) появляется на секунду и сразу пропадает, его даже не успеть прочитать, хотя бы еще секунду-две добавить, как в других фрагментах

10-2-3 Регрессия пик-плато

\url{https://youtu.be/zf3KEVbEslE} ок



\end{document}