\documentclass[12pt,a4paper]{article}
\usepackage[utf8]{inputenc}
\usepackage[russian]{babel}
\usepackage{url}
\usepackage{amsmath}
\usepackage{amsfonts}
\usepackage{amssymb}
\usepackage[left=1cm,right=1cm,top=1cm,bottom=2cm]{geometry}
\newcommand{\e}{\varepsilon}
\renewcommand{\b}{\beta}
\newcommand{\hb}{\hat{\b}}
\newcommand{\hs}{\hat{\sigma}}
\DeclareMathOperator*{\plim}{plim}
\usepackage{comment}
\usepackage{graphicx}
\begin{document}


10-1-1 Медианная регрессия

\url{https://youtu.be/nOmPHv9dpUk}

0:24 исправить заголовок слайда на <<Три сюжета>>

1:43 второй пункт исправить слово <<минизации>>, должно быть:

* При минимизации $Q(\hb_1,\hb_2)$ получаем состоятельные оценки $\hb_1$, $\hb_2$


10-1-2 Квантильная регрессия

\url{https://youtu.be/zecZye4a0_8}

2:17 вставить пробел перед <<и>>

* Нет явных формул для оценок коэффициентов и стандартных ошибок

3:47 убрать с графика подписи $\tau=0.1$ и $\tau=0.9$. Вместо них: прямо на графике, на той части, что закрашена в клеточку, написать $0.1$; на той части графика, что закрашена в полосочку, написать $0.8$; на той части, что не закрашена (справа от правой вертикальной линии), написать $0.1$. 

6:54 Сделать название слайда в одну строчку, горизонтальную ось графика вместо totsp назвать <<Общая площадь, м$^2$>>. Вертикальную ось вместо price назвать <<Цена квартиры, тыс. у.е.>>


10-1-3 Алгоритм случайного леса

\url{https://youtu.be/Ld_lhkfQ6bI}

2:05 продублировать справа сверху от дерева ту табличку, что висит в 1:16. И пусть она рядом с деревом висит

10-1-4 Пример построения регрессионного дерева

0:16 изменить название видеофрагмента (внизу на синей полосе) на <<Пример построения регрессионного дерева>>

\url{https://youtu.be/63fozZFGOG0}
10-1-5 \url{https://youtu.be/jsEx2Pom-1I}
10-1-6 \url{https://youtu.be/b5XvY97nMq0}
10-1-7 \url{https://youtu.be/6w4bRrAjDvs}
10-1-8 \url{https://youtu.be/vJKMxb3BJYw}
10-1-9 \url{https://youtu.be/c5BNpk_4o1g}


10-2-1 \url{https://youtu.be/hec3_Xu9Bfo}
10-2-2 \url{https://youtu.be/VoMy8Qt_AVk}
10-2-3 \url{https://youtu.be/zf3KEVbEslE}



\end{document}