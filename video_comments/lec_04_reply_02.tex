\documentclass[12pt,a4paper]{article}
\usepackage[utf8]{inputenc}
\usepackage[russian]{babel}
\usepackage{url}
\usepackage{amsmath}
\usepackage{amsfonts}
\usepackage{amssymb}
\usepackage[left=2cm,right=2cm,top=2cm,bottom=2cm]{geometry}
\renewcommand{\b}{\beta}
\newcommand{\s}{\sigma}
\newcommand{\hb}{\hat{\b}}
\newcommand{\hs}{\hat{\s}}

\begin{document}
4-1-1 Определение мультиколлинеарности

\url{http://www.youtube.com/watch?v=hw_3xlEANTY}

3:54 добавить надпись:

Способ борьбы со строгой мультиколлинеарностью:

Правильно включить в модель дамми-переменные

9:57 формула для $se^2$ написана неправильно, скобки должны быть на уровне $se$:

$se^2(\hb_j)=VIF_j \frac{\hs^2}{TSS_j}$

10:19 исправить формулу на

$sCorr(x,z)=\frac{\sum (x_i - \bar{x}) (z_i - \bar{z}) / (n-1) }{\sqrt{sVar(x) \cdot sVar(z)}}$

10:36 вторая появляющаяся формула должна быть исправлена на: (вместо $\widehat{Corr}...$):

* $sCorr(x,z)>0.9$    

4-1-2 Что поделать с мультиколлинеарностью?

\url{http://youtu.be/fUWiu3VosFs}

0:08 убрать слова <<кафедра публичной политики>>

2:05 опечатка в  заголовке слайда (пропущено ё) <<Жертвуем несмещённостью, чтобы снизить дисперсию>>

2:16 слово <<коэффициентов>> (в конце первого пункта) должно быть написано слитно

4-1-3 

\url{http://youtu.be/CGKQQn2JZRE}

0:16 изменить название фрагмента на <<Ридж и LASSO регрессия>> 

0:23 верхний заголовок исправить на <<Ридж и LASSO регрессия>> (без слова <<Пример>>)

4-1-4

\url{http://youtu.be/Q33zdzVaMjY}

0:08 убрать слова <<кафедра публичной политики>>

2:29 исправить на: (разница в конце строки!)

* переменная $pc_1$ имеет максимальную выборочную дисперсию $sVar(pc_1)$

3:04 исправить на: (разница в конце строки!)

* переменная $pc_2$ некоррелирована с $pc_1$ и имеет максимальную $sVar(pc_2)$

3:33 исправить на: (разница в конце строки!)

* переменная $pc_3$ некоррелирована с $pc_1$, $pc_2$ и имеет максимальную $sVar(pc_3)$

* и т.д.

4:32 сейчас надписи накладываются и становятся нечитабельными! должно быть:

Упрощенно:

Первая главная компонента --- математика

Вторая главная компонента --- биология

4-1-5 ок

\url{http://youtu.be/kT_oKuG2org}

4-1-6 

\url{http://youtu.be/nQobN0fO07I}

0:36 ниже второй формулы добавить многоточие (когда произношу слова <<и так далее>>)

0:42 исправить формулу на

$sCorr(pc_j, pc_m)=0$

0:52 исправить формулу на
\[
sVar(x_1)+ sVar(x_2) + \ldots + sVar(x_k) =
sVar(pc_1)+ sVar(pc_2) + \ldots + sVar(pc_k)
\]

6:52 выровнить строку (сейчас $pc_1$ и $pc_2$ ниже основной части строки)

Шаг 1. Найти главные компоненты: $pc_1$, $pc_2$

4-2-1ок

\url{http://youtu.be/pFu9b7Gsaas}

4-2-2 ок

\url{http://youtu.be/ej8iJvk1Xpk}

4-2-3 ок

\url{http://youtu.be/lbYjzdbNsHE}

4-2-4 ок

\url{http://youtu.be/VHG5cL7BUG8}



\end{document}