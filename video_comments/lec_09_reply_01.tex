\documentclass[12pt,a4paper]{article}
\usepackage[utf8]{inputenc}
\usepackage[russian]{babel}
\usepackage{url}
\usepackage{amsmath}
\usepackage{amsfonts}
\usepackage{amssymb}
\usepackage[left=1cm,right=1cm,top=1cm,bottom=2cm]{geometry}
\newcommand{\e}{\varepsilon}
\renewcommand{\b}{\beta}
\newcommand{\hb}{\hat{\b}}
\newcommand{\hs}{\hat{\sigma}}
\DeclareMathOperator*{\plim}{plim}
\usepackage{comment}
\usepackage{graphicx}
\begin{document}


9-1-1 http://youtu.be/HV1T_0WLg4o
2:29 дополнить появляющуюся формулу, чтобы вышло:
E(βˆ|X) = β, E(βˆ) = β
3:48 дополнить появляющуюся формулу, чтобы вышло:
E(βˆ|X) ̸= β, E(βˆ) ̸= β
9-1-2 http://youtu.be/etgx423Pkas
переснять из-за необходимости добавить дополнительный случай
9-1-3 http://youtu.be/Ev3WhjRglqw
6:57 исправить фразу на
* МНК оценивает на сколько растёт yi при росте наблюдаемого x∗i , включающего ошибку, на
единицу
9-1-4 Пропущенная объясняющая переменная http://youtu.be/R9J9cniL7vk ok
9-1-5 Метод инструментальных переменных http://youtu.be/ujHYU7A-mCE
0:25 исправить формулу на (добавлено di)
yi = β1 + β2xi + β3diεi, Cov(xi, εi) ̸= 0
1:24 ниже заголовка слайда добавляем пункт:
* нельзя просто заменить проблемный регрессор на инструментальную переменную
3:38 не хватает маленького лекционного кусочка. Там должен быть слайд «Простейший случай
двухшагового МНК» с содержимым:
МНК:
Метод инструментальных переменных:
yi = β1 + β2xi + εi βˆOLS = sCov(x,y)
2 sV ar(x) βˆIV = sCov(z,y)
2 sCov(z,x)
Если этого маленького кусочка почему-то нет, надо будет его доснять. 9-1-6 http://youtu.be/guYIPBygpIk
2:24 добавить под картинкой ссылку:
Randall Munroe, https://xkcd.com/552/
4:29 исправить заголовок слайда на «Данные экспериментов»
4:46 исправить ошибку в таблице, в последней строке должно быть «4 - Орёл - 0 - 1»
7 http://youtu.be/WcqWVZ-p6rc
3:10 добавить точку после H0:
Есть ошибка, смещающая результат в пользу H0. Вениамин обрадуется результату и, вероятно,
не заметит ошибку
3:47 добавить точку после Ha:
Есть ошибка, смещающая результат в пользу Ha. Вениамин будет удивлен, трижды перепроверит
работу и найдёт ошибку
6:30 Стираем слайд, делаем новый заголовок слайда «Мораль» 6:33 под заголовком слайда добавляем пункт:
* Эндогенность — коррелированность регрессора с ошибкой
6:36 добавляем ниже пункт:
* Метод инструментальных переменных
6:40 добавляем ниже пункт:
* Статистическая взаимосвязь не означает причинно-следственной
1
9-2-1 Деление выборки на обучающую и тестовую
http://youtu.be/iZDLRtmVbog ок

9-2-2 http://youtu.be/xzf18sGeqUg
2:27-2:51 Сейчас строка, которую я печатаю, оказывается обрезанной и не видно, что происходит.
Надо показать невлезающую строку.
9-2-3 http://youtu.be/lorcIDfUzzw
2:00-3:05 Сейчас строка, которую я печатаю, оказывается обрезанной и не видно, что происходит.
Надо показать невлезающую строку.

\end{document}