\documentclass[12pt,a4paper]{article}\usepackage[]{graphicx}\usepackage[]{color}
%% maxwidth is the original width if it is less than linewidth
%% otherwise use linewidth (to make sure the graphics do not exceed the margin)
\makeatletter
\def\maxwidth{ %
  \ifdim\Gin@nat@width>\linewidth
    \linewidth
  \else
    \Gin@nat@width
  \fi
}
\makeatother

\definecolor{fgcolor}{rgb}{0.345, 0.345, 0.345}
\newcommand{\hlnum}[1]{\textcolor[rgb]{0.686,0.059,0.569}{#1}}%
\newcommand{\hlstr}[1]{\textcolor[rgb]{0.192,0.494,0.8}{#1}}%
\newcommand{\hlcom}[1]{\textcolor[rgb]{0.678,0.584,0.686}{\textit{#1}}}%
\newcommand{\hlopt}[1]{\textcolor[rgb]{0,0,0}{#1}}%
\newcommand{\hlstd}[1]{\textcolor[rgb]{0.345,0.345,0.345}{#1}}%
\newcommand{\hlkwa}[1]{\textcolor[rgb]{0.161,0.373,0.58}{\textbf{#1}}}%
\newcommand{\hlkwb}[1]{\textcolor[rgb]{0.69,0.353,0.396}{#1}}%
\newcommand{\hlkwc}[1]{\textcolor[rgb]{0.333,0.667,0.333}{#1}}%
\newcommand{\hlkwd}[1]{\textcolor[rgb]{0.737,0.353,0.396}{\textbf{#1}}}%

\usepackage{framed}
\makeatletter
\newenvironment{kframe}{%
 \def\at@end@of@kframe{}%
 \ifinner\ifhmode%
  \def\at@end@of@kframe{\end{minipage}}%
  \begin{minipage}{\columnwidth}%
 \fi\fi%
 \def\FrameCommand##1{\hskip\@totalleftmargin \hskip-\fboxsep
 \colorbox{shadecolor}{##1}\hskip-\fboxsep
     % There is no \\@totalrightmargin, so:
     \hskip-\linewidth \hskip-\@totalleftmargin \hskip\columnwidth}%
 \MakeFramed {\advance\hsize-\width
   \@totalleftmargin\z@ \linewidth\hsize
   \@setminipage}}%
 {\par\unskip\endMakeFramed%
 \at@end@of@kframe}
\makeatother

\definecolor{shadecolor}{rgb}{.97, .97, .97}
\definecolor{messagecolor}{rgb}{0, 0, 0}
\definecolor{warningcolor}{rgb}{1, 0, 1}
\definecolor{errorcolor}{rgb}{1, 0, 0}
\newenvironment{knitrout}{}{} % an empty environment to be redefined in TeX

\usepackage{alltt}
\usepackage[utf8]{inputenc}
\usepackage[russian]{babel}
\usepackage{url}
\usepackage{amsmath}
\usepackage{amsfonts}
\usepackage{amssymb}
\usepackage[left=1cm,right=1cm,top=1cm,bottom=2cm]{geometry}
\newcommand{\e}{\varepsilon}
\renewcommand{\b}{\beta}
\newcommand{\hb}{\hat{\b}}
\newcommand{\hs}{\hat{\sigma}}
\usepackage{graphicx}
\IfFileExists{upquote.sty}{\usepackage{upquote}}{}
\begin{document}

Глава 7, итерация 1


\url{http://www.youtube.com/watch?v=0BJTraRatFg&list=PLCFTZGvx_s3f_saaG5Hl-wC6uSJJbOWFM}

7-1-1 Суть метода максимального правдоподобия

\url{http://www.youtube.com/watch?v=ahu8Xt7LEP0}

1:53 Изменяем заголовок на <<Метод максимального правдоподобия>>

1:53 под заголовком появляется <<ML --- Maximum likelihood>>

1:55 добавляем надпись ниже:

* Есть неизвестный параметр $\theta$ (!!!монтажёру: $\theta$ и $\hat{\theta}$ всегда синим цветом)

1:58 добавляем надпись ниже:

* Хотим построить оценку $\hat{\theta}$

2:00 добавляем надпись ниже:

* В качестве оценки неизвестного параметра $\theta$ возьмём такое число $\hat{\theta}$, при котором вероятность имеющихся данных максимальна.

2:16 под заголовком сразу появляется пункта:

* Наблюдения, количества звонков: $y_1=0$, $y_2=1$, $y_3=2$, $y_4=0$.

* Модель для наблюдений:

\begin{tabular}{c|ccc}
$y_i$ & $0$ & $1$ & $2$ \\ 
\hline 
$P(.)$ & $p$ & $2p$ & $1-3p$ \\ 
\end{tabular} 

2:37 ниже таблицы появляется вопрос задачи:

Оцените $\hat{p}$ методом максимального правдоподобия

7-1-2

\url{http://www.youtube.com/watch?v=c-zRZ29bz1U}

0:23 переставить слова в появляющейся фразе, должно быть:

Для непрерывных случайных величин максимизируется плотность вероятности

1:47 исправить формулу (сделать предлог <<при>> чёрным цветом, заменить $x$ на $y$), должно получиться:

Модель: наблюдения независимы, $f(y)=\lambda e^{-\lambda y}$ при $y>0$.

1:55 добавляем надпись (сейчас она появляется, но позже)

* Найдите $\hat{\lambda}$

1:58 пропадает строка с игреками, надо её вернуть обратно


7-1-3 Построение доверительных интервалов

\url{http://www.youtube.com/watch?v=_KOOJaZ4Glw}

0:28 сделать <<при>> черным цветом:

* Состоятельны: $\hat{\theta}_{ML} \to \theta$ при $n\to \infty$

0:38 сделать <<при>> черным цветом:

* Асимптотически несмещены: $E(\hat{\theta}_{ML}) \to \theta$ при $n\to \infty$

0:56 исправить появляющийся пункт на:

* Асимптотически эффективны: 

Дисперсия $Var(\hat{\theta}_{ML})$ наименьшая среди асимптотически несмещенных оценок

1:11 дополнительной надписи не появляется (старые пока остаются)

1:38 исправляем первый пункт (под заголовком слайда) на:

* Оценки $\hat{\theta}_{ML}$  асимптотически нормальны

1:40 в появляющейся строке делаем <<при>> черным цветом

1:50 исправить в появляющейся строке букву $l$ на более каллиграфическую $\ell$ (а то больно она на $I$ похожа), и слово <<информация>> с маленькой буквы, должно быть:

$I$ --- информация Фишера, $I=-E\left( \ell ''(\theta) \right)$

2:36 снова используем более каллиграфическую $\ell$, должно быть:

Наблюдаемая информация Фишера: $\hat{I}=-\ell ''(\hat{\theta})$ 

2:59 во второй формуле сделать более каллиграфическую $\ell$, должно быть:

$se(\hat{\theta})=\sqrt{\widehat{Var}(\hat{\theta}_{ML})}=\sqrt{-(\ell ''(\hat{\theta}))^{-1}}$


7-1-4 Проверка гипотез. LR тест.

\url{http://www.youtube.com/watch?v=0BJTraRatFg}

0:24 добавить пропущенные двоеточия, отделить <<хотя>> от <<бы>>, должно быть:

$H_0$: Система из $q$ уравнений на неизвестные параметры

$H_a$: Хотя бы одно из $q$ условий не выполнено

0:47 в формуле сделать 0 нижним индексом:

$LR=2(l(\hat{\theta})-l(\hat{\theta}_{H_0})) \sim \chi^2_q$

7-1-5  Логит-модель

\url{http://www.youtube.com/watch?v=3RULyVUXV74}

%\url{http://www.youtube.com/watch?v=yKBeG7p2RzA}

0:45 зелёная линия, образующая угол слева --- убрать

1:10 под заголовком слайда появляется надпись

$y_i=\begin{cases}
1, y^*_i \geq 0 \\
0, y^*_i <0
\end{cases}$

* Скрытая переменная: $y^*_i=\beta_1 +\beta_2 x_i +\varepsilon_i$

1:18 ниже появляется надпись:

* Пробит-модель: $\varepsilon_i \sim N(0,1)$

1:27 ниже надписи <<Пробит-модель...>> появляется надпись (в формуле исправлено $x$ на $t$):

* Логит-модель: $\varepsilon_i \sim logistic$, $f(t)=e^{-t}/(1+e^{-t})^2$

1:35 ниже надписи <<Логит-модель...>> появляется надпись 

* Логистическое распределение похоже на $N(0,1.6^2)$


1:55 старые надписи стираем, заменяем заголовок слайда на 

Вероятность $P(y_i=1)$

1:56 под заголовком появляется начало формулы

$P(y_i=1)=P(y_i^* \geq 0) = P(\beta_1 + \beta_2 x_i + \e_i \geq 0) =$

2:10 формула продолжается ниже

$=P(-\e_i \leq \beta_1 + \beta_2 x_i  ) = P(\e_i \leq \beta_1 + \beta_2 x_i  )=F(\beta_1 + \beta_2 x_i )$


2:40 -- 8:58 -- вместо старого куска с доской поставить отснятый 1 апреля :)


9:18 исправить вторую формулу на:

\[
\ln \frac{P(y_i=1)}{P(y_i=0)}=\beta_1 +\beta_2 x_i
\]


7-1-6 Интерпретация коэффициентов в логит-модели. Отношение шансов и вероятности. (доска), 6:19

\url{http://www.youtube.com/watch?v=mSV_h2sigS8} ок

7-1-7 Предельные эффекты и прогнозирование

\url{http://www.youtube.com/watch?v=cjCls8z6gok}

1:04 опечатка в заголовке слайда, должно быть <<Интерпретация>>

3:35 появляется дополнительный второй пункт:

* Точечный прогноз вероятности: $\hat{P}(y_f=1)=F(\hat{y}^*_f)$

3:50 появляется два дополнительных пункта:

* Доверительный интервал для $E(\hat{y}^*_f)$:

\[
[\hat{y}^*_f-z_{cr}se(\hat{y}^*_f);\hat{y}^*_f+z_{cr}se(\hat{y}^*_f)]
\]

* Доверительный интервал для вероятности $P(y_f=1)$:

\[
[F(\hat{y}^*_f-z_{cr}se(\hat{y}^*_f));F(\hat{y}^*_f+z_{cr}se(\hat{y}^*_f))]
\]

!!!монтажеру: эти четыре пункта висятс с 3:50 до 4:07 (больше ничего в этот период не добавляется)


4:20 исправляем появляющийся пункт на:

* Логит-модель: $y^*_i=\beta_1+\beta_2 x_i + u_i$, 
где $u_i$ примерно $N(0,1.6^2)$

4:40 появляется дополнительный пункт:

* Логит-модель: $\frac{y^*_i}{1.6}=\frac{\beta_1}{1.6}+\frac{\beta_2}{1.6} x_i + \frac{u_i}{1.6}$,
где $ \frac{u_i}{1.6}$ примерно $N(0,1)$

4:43 увеличение не нужно, по смыслу лучше все формулы обозревать

4:57 появляется дополнительный пункт:

Пробит-модель: $y^*_i=\beta_1+\beta_2 x_i +\e_i$,
где $\e_i \sim N(0,1)$

5:05 появляется дополнительный пункт ниже всех:

* $\{y_i=1\} \Leftrightarrow \{y_i^*>0\} \Leftrightarrow \{y_i^*/1.6>0\}$ 

5:26 и далее --- отрезать --- чёрный экран!!!

7-1-8 Несуществование ML оценок. Заключение

\url{http://www.youtube.com/watch?v=O898OdxrXW4}

10:15 добавляем пункт

* Метод максимального правдоподобия. Позволяет получать оценки неизвестных параметров.

10:20 добавляем ниже пункт:

* Логит и пробит модели. Модели для зависимой переменой, принимающей значения $0$ и $1$.

10:30 добавляем ниже пункт:

* МНК не подходит для моделирования бинарной зависимой переменной

7-2-1 Графический анализ качественных переменных

\url{http://www.youtube.com/watch?v=_7cfL8C98tc}

3:38-3:49 удалить неудачный фрагмент (далее всё идет ок и я говорю те же слова, только без оговорок)


7-2-2 Оценка коэффициентов и прогнозирование скрытой переменной

\url{http://www.youtube.com/watch?v=MdwF-uRru9Q}

3:08 (примерно, где я в камеру смотрю) сюда вставить кусок этого фрагмента с доской (7:40 и до конца)

7:40 и до конца (весь кусок с доской) нужно вырезать и вставить в точку 3:08

7-2-3 Доверительный интервал для вероятности и LR-тест

0:16 исправить название фрагмента (на синей полосе внизу) на <<Доверительный интервал для вероятности и LR-тест>>

\url{http://www.youtube.com/watch?v=w_2OCd0tC5k}

2:48 --- 4:08 вырезать полностью

в конец вставить начало из 7-2-4 (от начала и до 2:34)

\newpage
7-2-4 Предельные эффекты

\url{http://www.youtube.com/watch?v=4hl0bAdp45E}

0:16 исправить название фрагмента (на синей полосе внизу) на <<Предельные эффекты>>

от начала и до 2:34 вырезать и вставить в конец 7-2-3
% сравнение моделей (LR тест)

% предельные эффекты 2:34

4:21--4:29 вырезать (там я говорю, что надо разместить слева скриншот из программы)

4:29 -- 9:20 на прозрачной доске слева надо разместить выдачу из софта (я рассказываю справа, а слева висит выдача):




\begin{knitrout}
\definecolor{shadecolor}{rgb}{0.969, 0.969, 0.969}\color{fgcolor}\begin{kframe}
\begin{alltt}
\hlkwd{maBina}\hlstd{(m_logit)}
\end{alltt}
\begin{verbatim}
##             effect error t.value p.value
## (Intercept)  0.821 0.091   8.979   0.000
## sexmale     -0.551 0.029 -18.673   0.000
## age         -0.008 0.001  -5.391   0.000
## pclass2nd   -0.264 0.046  -5.747   0.000
## pclass3rd   -0.485 0.047 -10.271   0.000
## fare         0.000 0.000   0.191   0.849
\end{verbatim}
\end{kframe}
\end{knitrout}

\begin{knitrout}
\definecolor{shadecolor}{rgb}{0.969, 0.969, 0.969}\color{fgcolor}\begin{kframe}
\begin{alltt}
\hlkwd{maBina}\hlstd{(m_logit,}\hlkwc{x.mean} \hlstd{=} \hlnum{FALSE}\hlstd{)}
\end{alltt}
\begin{verbatim}
##             effect error t.value p.value
## (Intercept)  0.528 0.059   8.979   0.000
## sexmale     -0.551 0.029 -18.673   0.000
## age         -0.005 0.001  -5.391   0.000
## pclass2nd   -0.264 0.046  -5.747   0.000
## pclass3rd   -0.485 0.047 -10.271   0.000
## fare         0.000 0.000   0.191   0.849
\end{verbatim}
\end{kframe}
\end{knitrout}

% 4:21 здесь появляется...
9:21-9:22 вырезать <<угу>>

7-2-5 ROC кривая

\url{http://www.youtube.com/watch?v=G2iKEg0QJ0c}

0:16 исправить название фрагмента (на синей полосе внизу) на <<ROC кривая>>




\end{document}
