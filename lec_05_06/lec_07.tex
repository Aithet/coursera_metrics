\documentclass[ignorenonframetext,]{beamer}
\usepackage{amssymb,amsmath}
\usepackage{ifxetex,ifluatex}
\usepackage{fixltx2e} % provides \textsubscript
\usepackage{lmodern}
\ifxetex
  \usepackage{fontspec,xltxtra,xunicode}
  \defaultfontfeatures{Mapping=tex-text,Scale=MatchLowercase}
  \newcommand{\euro}{€}
\else
  \ifluatex
    \usepackage{fontspec}
    \defaultfontfeatures{Mapping=tex-text,Scale=MatchLowercase}
    \newcommand{\euro}{€}
  \else
    \usepackage[T1]{fontenc}
    \usepackage[utf8]{inputenc}
      \fi
\fi
\IfFileExists{upquote.sty}{\usepackage{upquote}}{}
% use microtype if available
\IfFileExists{microtype.sty}{\usepackage{microtype}}{}

% Comment these out if you don't want a slide with just the
% part/section/subsection/subsubsection title:
\AtBeginPart{
  \let\insertpartnumber\relax
  \let\partname\relax
  \frame{\partpage}
}
\AtBeginSection{
  \let\insertsectionnumber\relax
  \let\sectionname\relax
  \frame{\sectionpage}
}
\AtBeginSubsection{
  \let\insertsubsectionnumber\relax
  \let\subsectionname\relax
  \frame{\subsectionpage}
}

\setlength{\parindent}{0pt}
\setlength{\parskip}{6pt plus 2pt minus 1pt}
\setlength{\emergencystretch}{3em}  % prevent overfull lines
\setcounter{secnumdepth}{0}
\usepackage[utf8]{inputenc}
\usepackage[russian]{babel}

\title{Лекция 5. Гетероскедастичность}

\begin{document}
\frame{\titlepage}

\begin{frame}{Метод максимального правдоподобия}

Для проверки гипотез мы предполагали условную гомоскедастичность ошибок:

\$ E(\varepsilon\_i \textbar{} X)=\sigma\^{}2 \$

Что произойдет если эта предпосылка будет нарушена?

\end{frame}

\begin{frame}{Разница между условной и безусловной
гетероскедастичностью}

\end{frame}

\begin{frame}{Проблема:}

Все остальные предпосылки классической модели со стохастическими
регрессорами для случайной выборки выполнены.

Мы используем прежние формулы:

$\hat{\beta}=(X'X)^{-1}X'y$

$\widehat{Var}(\hat{\beta})=$

В частности, $\widehat{Var}(\hat{\beta}_j)=\frac{\hat{\sigma}^2}{RSS_j}$

\end{frame}

\begin{frame}{Свойства для конечных выборок:}

Без требования нормальности $\varepsilon$

\begin{itemize}
\item
  Линейность по $y$
\item
  Условная несмещенность, $E(\hat{\beta}|X)=\beta$
\item
  (---) Оценки неэффективны
\end{itemize}

\end{frame}

\begin{frame}{С требованием нормальности $\varepsilon$}

\begin{itemize}
\item
  (---)
  $t=\frac{\hat{\beta}_j-\beta_j}{se(\hat{\beta}_j)} | X \sim t_{n-k}$
\item
  (---) $RSS/\sigma^2 |X \sim \chi^2_{n-k}$
\end{itemize}

\end{frame}

\begin{frame}{Асимптотические свойства:}

\end{frame}

\end{document}
