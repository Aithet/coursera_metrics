\documentclass[12pt,a4paper]{article}
\usepackage[utf8]{inputenc}
\usepackage[russian]{babel}

\usepackage{amsmath}
\usepackage{amsfonts}
\usepackage{amssymb}
\usepackage[left=2cm,right=2cm,top=2cm,bottom=2cm,landscape]{geometry}
\begin{document}


{\Huge

\begin{enumerate}

\item Гомоскедастичность

Для проверки гипотез мы предполагали условную гомоскедастичность ошибок:

$ E(\varepsilon_i^2 | X)=\sigma^2 $

Что произойдет если эта предпосылка будет нарушена?

\item Разница между условной и безусловной гетероскедастичностью

тут вставка

\item Когда логично ожидать гетероскедастичность?

* безусловной в случайной выборке не бывает

* условная присутствует почти всегда

* наличие <<размера>> объекта

\item В остальном всё ок

Все остальные предпосылки классической модели со стохастическими регрессорами для случайной выборки выполнены.

(тут пачка предпосылок)

\newpage
\item  Мы используем прежние формулы:

$\hat{\beta}=(X'X)^{-1}X'y$

$\widehat{Var}(\hat{\beta})=\frac{RSS}{n-k}(X'X)^{-1}$

В частности, $\widehat{Var}(\hat{\beta}_j)=\frac{\hat{\sigma}^2}{RSS_j}$


\item Свойства для конечных выборок:

Без требования нормальности $\varepsilon$

* Линейность по $y$

* Условная несмещенность, $E(\hat{\beta}|X)=\beta$

* (---) Оценки неэффективны

\newpage
\item  С требованием нормальности $\varepsilon$

* (---) $\frac{\hat{\beta}_j-\beta_j}{se(\hat{\beta}_j)} | X \sim t_{n-k}$

* (---) $\frac{RSS}{\sigma^2} |X \sim \chi^2_{n-k}$

* (---) $\frac{(RSS_R-RSS_{UR})/r}{RSS_{UR}/(n-k)} \sim F_{r,n-k}$

\item  Асимптотические свойства:

*  $\hat{\beta} \to \beta $

* $\frac{RSS}{n-k} \to \sigma^2 $

* (---) $\frac{\hat{\beta}_j-\beta_j}{se(\hat{\beta}_j)} \to N(0,1)$

* (---) $\frac{RSS_R-RSS_{UR}}{RSS_{UR}/(n-k)} \to \chi^2_r$

\newpage
\item Мораль:

* Сами $\hat{\beta}$ можно интерпретировать и использовать

* Стандартные ошибки $se(\hat{\beta}_j)$ несостоятельны

* Не можем строить доверительные интервалы

\item Что делать

* Исправить стандартные ошибки! 


* Вместо $\widehat{Var}(\hat{\beta})=\frac{RSS}{n-k}(X'X)^{-1}$ использовать

* $\widehat{Var}(\hat{\beta})=(X'X)^{-1}X'\hat{\Omega}X(X'X)^{-1}$

* Уайт, 1980, HC0: 

$\hat{\Omega}=diag( \hat{\varepsilon}_1^2, \ldots, \hat{\varepsilon}_n^2 )$

* Современный вариант, HC3: 

$\hat{\Omega}=diag( \frac{\hat{\varepsilon}_1^2}{(1-h_{11})^2}, \ldots, \frac{\varepsilon_n^2}{(1-h_{nn})^2} )$

\newpage
\item С практической точки зрения:

* Новая формула, $se_{HC}(\hat{\beta}_j)$

* R: $vcovHC(model)$

* С ней жизнь прекрасна!

$\frac{\hat{\beta}_j-\beta_j}{se_{HC}(\hat{\beta}_j)} \to N(0,1)$

\item Практичный подход!

* Как только есть случайная выборка и объекты могут быть разного <<размера>>, использовать $se_{HC}(\hat{\beta}_j)$ для проверки гипотез

\newpage
\item Обнаружение гетероскедастичности

* Графическое

\item Формальные тесты на гск

* Тест Уайта

* Асимптотический, не требуется нормальность остатков

\item Тест Уайта  начало

* Оценить основную регрессию, получить $\hat{\varepsilon}_i$

* Оценить вспомогательную регрессию:

$\hat{\varepsilon}^2_i = \gamma_1 + \gamma_2 z_{i2} + \ldots + \gamma_{i,k} z_{im}+ u_i$

$z_{i2}$, \ldots, $z_{im}$ --- факторы, определяющие форму гетероскедастичности

Посчитать $LM=nR^2_{aux}$

\newpage
\item Тест Уайта продолжение

При верной $H_0$ об условной гомоскедастичности

$H_0$: $E(\varepsilon^2_i|X)=\sigma^2$

$LM \sim \chi^2_{m-1}$, где $m$ --- число параметров во вспомогательной регрессии

По умолчанию во вспомогательной регрессии берут исходные регрессоры, их квадраты и попарные произведения

\item вставка тест уайта

По 200 киоскам мороженого и  исследователь оценил зависимость спроса (q) от цены (p), разнообразия ассортимента (a) и удаленности от метро (d).

Какой регрессор скорее всего влияет на условную дисперсию ошибок?

Исследователь провел классический тест Уайта и получил $R^2_{aux}=0.2$.

Как выглядит вспомогательная регрессия?

Имеет ли место условная гетероскедастичность?

\newpage
\item Тест Голдфельда-Квандта

* Есть переменная, от которой может зависеть условная дисперсия ошибок

* Требуется нормальность ошибок

* Тест подходит для малых выборок

\item Процедура теста ГК

* Сортируем наблюдения по предполагаемому убыванию условной дисперсии

* Выкидываем часть наблюдений посередине (20\%)

* оцениваем исходную модель отдельно по первым и по последним наблюдениям

* Считаем $F=\frac{RSS_1/(n_1-k)}{RSS_2/(n_2-k)}$

\newpage
\item Тест ГК-2

* При верной $H_0$ об условной гомоскедастичности

$H_0$: $E(\varepsilon^2_i|X)=\sigma^2$

$F\sim F_{n_1-k,n_2-k}$


\item Вставка

По 200 киоскам мороженого и  исследователь оценил зависимость спроса (q) от цены (p), разнообразия ассортимента (a) и удаленности от метро (d).

Чтобы проверить наличие гетероскедастичности исследователь оценил эту модель отдельно по 80 самым удаленным от метро киоскам, получил, $RSS_2=120$. По 80 самым близки к метро киоскам, получил, $RSS_1=210$. 

\newpage
\item Эффективность?

* Да, надо смириться с тем, что оценки неэффективны

* Мы довольны несмещенностью, состоятельностью и возможностью проверять гипотезы

* Для получения эффективных оценок нужно точно понимать как устроена гетероскедастичность. Это большая редкость.


\item Вставка 

Если бы мы знали как устроена гетероскедастичность...


\item Мораль

* Мы рассмотрели ситуацию нарушения предпосылки условной гомоскедастичности

* Почти всегда нарушена
 
* Неприятность мелкая, мы переживем использая состоятельные стандартные ошибки







 

 




\end{enumerate}





} % скобка от хьюго
\end{document}