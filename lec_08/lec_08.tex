\documentclass[ignorenonframetext,]{beamer}
\usetheme{CambridgeUS}
\usepackage{amssymb,amsmath}
\usepackage{ifxetex,ifluatex}
\usepackage{fixltx2e} % provides \textsubscript
\usepackage{lmodern}
\ifxetex
  \usepackage{fontspec,xltxtra,xunicode}
  \defaultfontfeatures{Mapping=tex-text,Scale=MatchLowercase}
  \newcommand{\euro}{€}
\else
  \ifluatex
    \usepackage{fontspec}
    \defaultfontfeatures{Mapping=tex-text,Scale=MatchLowercase}
    \newcommand{\euro}{€}
  \else
    \usepackage[T1]{fontenc}
    \usepackage[utf8]{inputenc}
      \fi
\fi
\IfFileExists{upquote.sty}{\usepackage{upquote}}{}
% use microtype if available
\IfFileExists{microtype.sty}{\usepackage{microtype}}{}

% Comment these out if you don't want a slide with just the
% part/section/subsection/subsubsection title:
\AtBeginPart{
  \let\insertpartnumber\relax
  \let\partname\relax
  \frame{\partpage}
}
\AtBeginSection{
  \let\insertsectionnumber\relax
  \let\sectionname\relax
  \frame{\sectionpage}
}
\AtBeginSubsection{
  \let\insertsubsectionnumber\relax
  \let\subsectionname\relax
  \frame{\subsectionpage}
}

\setlength{\parindent}{0pt}
\setlength{\parskip}{6pt plus 2pt minus 1pt}
\setlength{\emergencystretch}{3em}  % prevent overfull lines
\setcounter{secnumdepth}{0}
\usepackage[russian]{babel}

\title{Эконометрика. Лекция 8. Модели временных рядов}

\begin{document}
\frame{\titlepage}

\begin{frame}{Временные ряды:}

\begin{itemize}
\itemsep1pt\parskip0pt\parsep0pt
\item
  Многомерные
\end{itemize}

(тут табличка)

\begin{itemize}
\itemsep1pt\parskip0pt\parsep0pt
\item
  Одномерные
\end{itemize}

\end{frame}

\begin{frame}{Одномерный временной ряд}

Временной ряд --- последовательность случайных величин

\[
X_1, X_2, X_3, \ldots
\]

\end{frame}

\begin{frame}{Без предположений невозможно прогнозировать}

1, 2, 3, 4, 5, ?

(потом появляется правильный ответ: 42)

\end{frame}

\begin{frame}{Базовое предположение --- стационарность}

Временной ряд называется стационарным, если:

\begin{itemize}
\itemsep1pt\parskip0pt\parsep0pt
\item
  $E(X_1)=E(X_2)=E(X_3)=\ldots$
\item
  $Var(X_1)=Var(X_2)=Var(X_3)=\ldots$
\item
  $Cov(X_1,X_2)=Cov(X_2,X_3)=Cov(X_3,X_4)=\ldots$
\item
  $Cov(X_1,X_3)=Cov(X_2,X_4)=Cov(X_3,X_5)=\ldots$
\item
  \ldots
\end{itemize}

\end{frame}

\begin{frame}{Графики}

Пример с меняющимся $E(X_t)$

Пример с меняющейся $Var(X_t)$

Пример стационарного

\end{frame}

\begin{frame}{Предпосылки коротко:}

Временной ряд называется стационарным, если:

\begin{itemize}
\itemsep1pt\parskip0pt\parsep0pt
\item
  $E(X_t)=const$
\item
  $Cov(X_t,X_{t-k})=\gamma_k$
\end{itemize}

\end{frame}

\begin{frame}{Самый простой пример --- белый шум}

Ряд $\varepsilon_t$ --- белый шум, если:

\begin{itemize}
\itemsep1pt\parskip0pt\parsep0pt
\item
  $E(\varepsilon_t)=0$
\item
  $Var(\varepsilon_t)=\sigma^2$
\item
  $Cov(\varepsilon_t,\varepsilon_{t-k})=0$
\end{itemize}

\end{frame}

\end{document}
