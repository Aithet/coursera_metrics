\documentclass[ignorenonframetext,]{beamer}
\usepackage{amssymb,amsmath}
\usepackage{ifxetex,ifluatex}
\usepackage{fixltx2e} % provides \textsubscript
\usepackage{lmodern}
\ifxetex
  \usepackage{fontspec,xltxtra,xunicode}
  \defaultfontfeatures{Mapping=tex-text,Scale=MatchLowercase}
  \newcommand{\euro}{€}
\else
  \ifluatex
    \usepackage{fontspec}
    \defaultfontfeatures{Mapping=tex-text,Scale=MatchLowercase}
    \newcommand{\euro}{€}
  \else
    \usepackage[T1]{fontenc}
    \usepackage[utf8]{inputenc}
      \fi
\fi
\IfFileExists{upquote.sty}{\usepackage{upquote}}{}
% use microtype if available
\IfFileExists{microtype.sty}{\usepackage{microtype}}{}
\usepackage{longtable,booktabs}
\usepackage{caption}
% These lines are needed to make table captions work with longtable:
\makeatletter
\def\fnum@table{\tablename~\thetable}
\makeatother

% Comment these out if you don't want a slide with just the
% part/section/subsection/subsubsection title:
\AtBeginPart{
  \let\insertpartnumber\relax
  \let\partname\relax
  \frame{\partpage}
}
\AtBeginSection{
  \let\insertsectionnumber\relax
  \let\sectionname\relax
  \frame{\sectionpage}
}
\AtBeginSubsection{
  \let\insertsubsectionnumber\relax
  \let\subsectionname\relax
  \frame{\subsectionpage}
}

\setlength{\parindent}{0pt}
\setlength{\parskip}{6pt plus 2pt minus 1pt}
\setlength{\emergencystretch}{3em}  % prevent overfull lines
\setcounter{secnumdepth}{0}
\usepackage[utf8]{inputenc}
\usepackage[russian]{babel}

\title{Лекция 7. Максимально правдоподобно о логит и пробит-моделях}

\begin{document}
\frame{\titlepage}

\begin{frame}{Метод максимального правдоподобия}

Наблюдения: вижу работающий фонтан

Гипотеза 1: фонтан работает каждый день

Гипотеза 2: фонтан включают раз в году

\end{frame}

\begin{frame}{Правдоподобие. Более формально}

Метод максимального правдоподобия (ML --- Maximum Likelihood)

В качестве оценки неизвестного параметра $\theta$ возьмем такое число
$\hat{\theta}$, при котором вероятность имеющихся данных максимальна.

\end{frame}

\begin{frame}{Пример задачи}

Наблюдения: $y_1=0$, $y_2=1$, $y_3=2$, $y_4=0$.

Модель: наблюдения независимы,

\begin{longtable}[c]{@{}rrrr@{}}
\toprule\addlinespace
$y$ & 0 & 1 & 2
\\\addlinespace
\midrule\endhead
Вероятность & $p$ & $2p$ & $1-3p$
\\\addlinespace
\bottomrule
\end{longtable}

\end{frame}

\begin{frame}{Решаем задачу у чудо-доски}

\end{frame}

\begin{frame}{Правдоподобие. Непрерывный случай}

Для непрерывных случайных величин максимизируется плотность вероятности

Для независимых наблюдений:
$f(y_1,y_2,...,y_n|\theta)=f(y_1|\theta)\cdot f(y_2|\theta)\cdot ...\cdot f(y_n|\theta)=\prod f(y_i|\theta)$

Трюк с логарифмированием:
$l(\theta)=\ln\left( \prod f(y_i|\theta) \right) = \sum \ln f(y_i|\theta)$

\end{frame}

\begin{frame}{Задача 2.}

100 наблюдений: $y_1=1.1$, $y_2=2.7$, \ldots{}, $y_{100}=1.5$.

Сумма, $\sum y_i=200$.

Модель: наблюдения независимы, $f(y)=\lambda e^{-\lambda x}$ при $x>0$.

Найдите $\hat{\lambda}$

\end{frame}

\begin{frame}{Решаем задачу 2 чудо-доска.}

\end{frame}

\begin{frame}{ML --- это хорошо!}

ML оценки:

\begin{itemize}
\itemsep1pt\parskip0pt\parsep0pt
\item
  Состоятельны: $\hat{\theta}_{ML} \to \theta$ при $n\to \infty$
\item
  Асимптотически несмещены: $E(\hat{\theta}_{ML}) \to \theta$ при
  $n\to \infty$
\item
  Асимптотически эффективны:
\end{itemize}

$Var(\hat{\theta}_{ML})$ наименьшая среди асимптотически несмещенных

\end{frame}

\begin{frame}{ML --- это нормально!}

\begin{itemize}
\itemsep1pt\parskip0pt\parsep0pt
\item
  Асимптотически нормальны:
\end{itemize}

$\hat{\theta}_{ML} \sim N(\theta, I^{-1})$ при $n>>0$

$I$ --- информация Фишера, $I=-E\left( l''(\theta) \right)$

В многомерном случае: $I=-E( H )$, $H$ --- матрица Гессе

\end{frame}

\begin{frame}{ML оценка как случайная величина}

Среднее: $E(\hat{\theta}_{ML}) \approx \theta$, дисперсия:
$Var(\hat{\theta}_{ML}) \approx I^{-1}$

Оценка дисперсии: $\widehat{Var}(\hat{\theta}_{ML})=\hat{I}^{-1}$

Наблюдаемая информация Фишера $\hat{I}=-l''(\hat{\theta})$

\end{frame}

\begin{frame}{Доверительный интервал}

Доверительный интервал:

\[
\theta \in [\hat{\theta}_{ML}-z_{cr}se(\hat{\theta});\hat{\theta}_{ML}+z_{cr}se(\hat{\theta})],
\]

$se(\hat{\theta}))=\sqrt{\widehat{Var}(\hat{\theta}_{ML})}=\sqrt{-(l''(\hat{\theta}))^{-1}}$

\end{frame}

\begin{frame}{Продолжение задачи у чудо-доски}

Постройте 95\%-ый доверительный интервал для $\theta$.

\end{frame}

\begin{frame}{Проверка гипотез}

$H_0$: Система из $q$ уравнений на неизвестные параметры

$H_a$: Хотя бы одно из $q$ условий не выполнено

Тест отношения правдоподобия (Likelihood Ratio, LR):

\[
LR=2(l(\hat{\theta})-l(\hat{\theta}_{H0})) \sim \chi^2_q
\]

\end{frame}

\begin{frame}{Продолжение задачи у чудо-доски}

Проверьте гипотезу $H_0$: $\lambda=1$.

\end{frame}

\begin{frame}{Логит и пробит-модели}

Бинарная объясняемая переменная: $y_i \in \{0,1\}$.

Скрытая ненаблюдаемая переменная:
$y^*_i=\beta_1 +\beta_2 x_i +\varepsilon_i$.

$y_i=\begin{cases} 1, y^*_i \geq 0 \\ 0, y^*_i <0 \end{cases}$

\end{frame}

\begin{frame}{Разница логит-пробит}

Логит-модель: $\varepsilon_i \sim logistic$, $f(t)=e^{-x}/(1+e^{-x})^2$

Пробит-модель: $\varepsilon_i \sim N(0,1)$.

Логистическое похоже на $N(0,1.6^2)$

\end{frame}

\begin{frame}{Вероятность}

\begin{multline}
P(y_i=1)=P(y^*_i\geq 0)=P(\beta_1 +\beta_2 x_i +\varepsilon_i \geq 0)=\\
=P( -\varepsilon_i \leq \beta_1 +\beta_2 x_i  ) = 
P( \varepsilon_i \leq \beta_1 +\beta_2 x_i  ) = \\
=F(\beta_1 +\beta_2 x_i) = \int_{-\infty}^{\beta_1+\beta_2 x_i} f(t) dt
\end{multline}

\end{frame}

\begin{frame}{Упражнение.}

Для логит-модели найдите $P(y_i=1)$, $\ln P(y_i=1)/P(y_i=0)$

\end{frame}

\begin{frame}{Чудо-Доска}

\end{frame}

\begin{frame}{Логарифмическое отношение шансов}

Для логит-модели:

\[
P(y_i=1)=\frac{1}{1+exp(-(\beta_1+\beta_2 x_i))}
\]

\[
\ln P(y_i=1)/P(y_i=0)=\beta_1 +\beta_2 x_i
\]

\end{frame}

\begin{frame}{Функция правдоподобия}

Наблюдения: $y_1=1$, $y_2=0$, \ldots{}

Модель: логит.

Функция правдоподобия: \[ 
P(y_1=1, y_2=0, ...)=P(y_1=1)\cdot P(y_2=0)\cdot ...
\]

\end{frame}

\begin{frame}{Интерпретация}

Коэффициенты плохо интерпретируемы

Предельный эффект --- производная вероятности:

\[
\frac{dP(y=1)}{dx}=\frac{dF(\beta_1+\beta_2 x)}{dx}=\beta_2 \cdot f(\beta_1+\beta_2x)
\]

Зависит от $x$ (!)

\end{frame}

\begin{frame}{Два средних предельных эффекта:}

Средний предельный эффект по наблюдениям:

\[
\frac{\sum \beta_2 \cdot f(\beta_1+\beta_2 x_i)}{n}
\]

Предельный эффект для среднего наблюдения:

\[
\beta_2 \cdot f(\beta_1+\beta_2 \bar{x})
\]

\end{frame}

\begin{frame}{Прогнозирование}

Прогноз скрытой переменной:
$\hat{y}^*_f=\hat{\beta}_1+\hat{\beta}_2 x_f$

Доверительный интервал для $E(\hat{y}^*_f)$:

\[
[\hat{y}^*_f-z_{cr}se(\hat{y}^*_f);\hat{y}^*_f+z_{cr}se(\hat{y}^*_f)]
\]

Переход к $\hat{P}(y_f=1)=F(y^*_f)$

\end{frame}

\begin{frame}{Разница логит-пробит на практике}

Коэффициенты логит/пробит отличаются в $\sim 1.6$ раза:

Логит (примерно): $y^*_i=\beta_1+\beta_2 x_i +N(0,1.6^2)$

\[
\frac{y^*_i}{1.6}=\frac{\beta_1}{1.6}+\frac{\beta_2}{1.6} x_i +N(0,1)
\]

Пробит: $y^*_i=\beta_1+\beta_2 x_i +N(0,1)$

\end{frame}

\begin{frame}{Проблема логит-пробит моделей}

``Идеальное прогнозирование'':

\begin{longtable}[c]{@{}rrr@{}}
\toprule\addlinespace
$y_1=0$ & $y_2=0$ & $y_3=1$
\\\addlinespace
\midrule\endhead
$x_1=1$ & $x_2=2$ & $x_3=3$
\\\addlinespace
\bottomrule
\end{longtable}

ML оценки не существуют!

\end{frame}

\begin{frame}{Объяснение с помощью ЧД}

\end{frame}

\begin{frame}{Проблема логит-пробит моделей}

Нередко возникает при большом количестве дамми-регрессоров

Признаки: не сходится ML,

R: ``fitted probabilities numerically 0 or 1 occurred''

Решения: регуляризация, байесовский подход

\end{frame}

\end{document}
